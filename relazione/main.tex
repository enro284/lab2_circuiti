\documentclass[12pt]{article}
\usepackage[italian]{babel}
\usepackage[a4paper, margin=2cm]{geometry}
\usepackage{fontspec}
\setmainfont{calibri}
\usepackage{titlesec}
\titleformat{\section}
  {\normalfont\normalsize\itshape}{\thesection.}{1em}{}
\titlespacing{\section}{0pt}{1em}{0pt}

\usepackage{cleveref}

\title{TITOLO}
\author{Enrico Barbuio \\ 0001117553 \and Giacomo Cicala} 
\date{data}

\begin{document}
\maketitle  
\section*{Abstract}
Lunghezza massima 10 righe, riassumere lo scopo dell’esperimento e i risultati principali
(osservazioni qualitative e risultati numerici). L’abstract non deve contenere riferimenti a quanto
riportato nel testo (figure, equazioni o tabelle) e deve essere auto – esplicativo.

\section*{Introduzione}
Descrivere i concetti fisici principali e gli obiettivi dell’esperienza. Per l’esperienza sui circuiti
descrivere il circuito progettato, le espressioni algebriche per le grandezze fisiche misurate, il
comportamento atteso; analogamente per ottica. L’introduzione non è un riassunto, non riportare
i risultati.

\section*{Apparato sperimentale e svolgimento}
Descrivere l’apparato utilizzato. Allegare una foto o lo schema dell’apparato sperimentale
(sarà la Fig. 1); la foto deve permettere di identificare chiaramente i componenti principali
(eventualmente usare delle frecce e delle etichette per evidenziarli). Riassumere lo svolgimento
delle misure e motivare la scelta dei parametri di acquisizione (per esempio: l’intervallo di frequenze
e la frequenza di acquisizione).
\section*{Risultati e discussione}
Riportare i risultati più rappresentativi in forma grafica oppure come foto delle osservazioni
sull’oscilloscopio analogico (se utilizzato). Non è necessario riportare tutti i dati. Commentare
qualitativamente gli andamenti delle grandezze fisiche riportati in forma grafica e/o le forme di riga
osservate. In tutti i grafici gli assi devono essere chiaramente etichettati e le unità di misura devono
essere incluse. Quando le incertezze sono note e sono visibili sulla scala utilizzata rappresentarle sul
grafico come barre di errore. Porre particolare attenzione alla leggibilità, utilizzando caratteri
sufficientemente grandi.
Descrivere come sono stati elaborati i dati e riportare i risultati numerici (miglior stima ed
incertezza), in forma tabellare se necessario. Commentare i valori ottenuti. Non è necessario
riportare il calcolo esplicito delle incertezze (eventualmente usare una appendice), ma è importante
segnalare se si tratta di risoluzione strumentale, di errore casuale oppure di errore sistematico.
\section*{Conclusioni}
Conclusioni finali, particolarmente importanti nel caso di risultati anomali.
\end{document}