\documentclass[12pt,italian]{article}
\usepackage[italian]{babel}
\usepackage[a4paper, margin=2cm]{geometry}
\usepackage{amsmath}
\usepackage{circuitikz}
\usepackage{caption}
\usepackage{cleveref}
\usepackage{amsfonts}

\newcommand{\prof}[1]{\textcolor{blue}{#1}}
\newcommand{\err}[1]{\textcolor{red}{#1}}

%\usepackage{fontspec} \setmainfont{calibri}
%\usepackage{titlesec}
%\titleformat{\section}
%{\normalfont\normalsize\itshape}{\thesection.}{1em}{}
%\titlespacing{\section}{0pt}{1em}{0pt}


\title{TITOLO}
\author{Enrico Barbuio \\ 0001117553 \and Giacomo Cicala \\ 0001122965} 
\date{data}

\begin{document}
\maketitle
\section*{Abstract}
\prof{ Lunghezza massima 10 righe, riassumere lo scopo dell'esperimento e i
	risultati principali (osservazioni qualitative e risultati numerici).
	L'abstract non deve contenere riferimenti a quanto riportato nel testo
	(figure, equazioni o tabelle) e deve essere auto - esplicativo. }

L'esperimento ha avuto come obiettivo la progettazione e la realizzazione, su
breadboard Elvis II, di un crossover RLC a tre vie (woofer, midrange, tweeter),
corredato dallo studio analitico del comportamento in regime sinusoidale e dal
confronto con le misure sperimentali. Dal punto di vista qualitativo si è
osservato che la risposta in ampiezza del woofer decresce verso zero alle alte
frequenze, quella del tweeter cresce \err{fino al valore del generatore} e il
midrange presenta un picco alla risonanza; le misure di fase confermano
l'inversione di $ \pm 90^\circ $ fuori banda e la coerenza in banda di taglio.
Il valore teorico della frequenza di crossover tra woofer e tweeter, calcolato
dalla relazione analitica, è risultato

\begin{equation*}
	f_{c} = (\cdots \pm \cdots)
	\ kHz
\end{equation*},

\noindent
mentre la misura sperimentale ha fornito

\begin{equation*}
	f_{c} = (\cdots \pm \cdots) \ kHz
\end{equation*}.

\noindent
La frequenza di risonanza del midrange attesa era

\begin{equation*}
	f_{r} = (\cdots \pm \cdots) \ kHz
\end{equation*},

\noindent
e quella misurata
\begin{equation*}
	f_{r} = (\cdots \pm \cdots) \ kHz
\end{equation*}.
\err{sistemare formattazione eq. con virgole e punti, aggiungere anche differenza di fase al crossover?}

\section*{Introduzione}
\prof{ Descrivere i concetti fisici principali e gli obiettivi dell'esperienza.
	Per l'esperienza sui circuiti descrivere il circuito progettato, le
	espressioni algebriche per le grandezze fisiche misurate, il comportamento
	atteso; analogamente per ottica. L'introduzione non è un riassunto, non
	riportare i risultati.}

L'esperimento svolto in laboratorio è incentrato sulla progettazione di un
circuito RLC sulla breadboard di Elvis II, sullo studio analitico del
comportamento atteso e sul confroto di questo con le misure effettuate. Il
circuito scelto è un crossover a tre vie, un circuito filtrante a regime
sinusoidale dato dalla combinazione di un filtro passa-basso (woofer), un
filtro passa-alto (tweeter) e un filtro passa-banda (midrange) che suddivide il
segnale audio in tre bande di frequenza distinte (bassi, alti, medi), al fine
di ottimizzare la riproduzione sonora per tutta la gamma di frequenze.

Dalla legge di Ohm e dalle leggi di Kirchhoff simboliche per tensioni e
correnti (\err{inserire eq. fondamentali inline oppure fare appendice dove si
	ricava tutto?}), utilizzando il metodo dei fasori, possiamo ricavare
l'andamento dell'ampiezza della differenza di potenziale ai capi del generatore
e ai capi delle resistenze di woofer, medi e tweeter [\ref{sec:tensioni} \err{farlo o no?}]

\begin{equation}
	\left| \mathbf{V_{g}}(\omega) \right| = \left| \frac{\mathbb{Z}_{load}(\omega)}
	{R_{g}+\mathbb{Z}_{load}(\omega)}\right| V_{0}
	\label{eq:Vg}
\end{equation}

\begin{equation}
	\left| \mathbf{V_{w}}(\omega) \right| = \left| \frac{R_{w}}
	{\mathbb{Z}_{w}(\omega)}\right|\left| \mathbf{V_{g}}(\omega) \right|
	\label{eq:Vw}
\end{equation}

\begin{equation}
	\left| \mathbf{V_{m}}(\omega) \right| = \left| \frac{R_{m}}
	{\mathbb{Z}_{m}(\omega)}\right|\left| \mathbf{V_{g}}(\omega) \right|
	\label{eq:Vm}
\end{equation}

\begin{equation}
	\left| \mathbf{V_{t}}(\omega) \right| = \left| \frac{R_{t}}
	{\mathbb{Z}_{t}(\omega)}\right|\left| \mathbf{V_{g}}(\omega) \right|
	\label{eq:Vt}
\end{equation}
\\
e delle rispettive fasi (\err{decidere dove esplicitare e definire simboli}) [\ref{sec:fasi}]

\begin{equation}
	\phi_{w}(\omega) = Arg(\mathbf{V_{g}}(\omega)) - Arg(\mathbb{Z}_{w}(\omega))
\end{equation}

\begin{equation}
	\phi_{m}(\omega) = Arg(\mathbf{V_{g}}(\omega)) - Arg(\mathbb{Z}_{m}(\omega))
\end{equation}

\begin{equation}
	\phi_{t}(\omega) =  Arg(\mathbf{V_{g}}(\omega)) - Arg(\mathbb{Z}_{t}(\omega))
\end{equation}
\\
tutto in funzione della pulsazione.

Per quanto riguarda le ampiezze delle tensioni in funzione della frequenza, per
il woofer ci aspettiamo un andamento decrescente asintoticamente a zero, per il
tweeter ci aspettiamo un andamento crescente asintoticamente al generatore,
mentre per il midrange ci aspettiamo un picco alla frequenza di risonanza.
Invece, per quanto riguarda le fasi, ci aspettiamo che il woofer e il tweeter
siano in fase con il generatore rispettivamente a basse e alte frequenze, e che
siano in ritardo e in anticipo di fase di $90^\circ$ nel caso contrario; il
midrange, invece, ci aspettiamo che sia in fase con il generatore alla
frequenza di risonanza.

Da \eqref{eq:Vw} e \eqref{eq:Vt} è possibile ricavare il valore atteso della
frequenza di crossover [\ref{sec:crossFreq}], cioè la frequenza alla quale il
woofer e il tweeter hanno la tensione di ampiezza uguale.

\begin{equation}
	f_{c} = \frac{1}{2\pi}\sqrt{\frac{\sqrt{4 (R_{w} R_{t} L_{w})^2 + C_{t}^2(2
				R_{w} R_{Lw} R_{t}^2 + (R_{Lw} R_{t})^2)^2}}{2 C _{t} (L_{w} R_{t})^2 } -
		\frac{R_{w} R_{Lw}}{L_{w}^2} - \frac{R_{Lw}^2}{2 L_{w}^2}}
	\label{eq:fc}
\end{equation}

\noindent
Inoltre, da \eqref{eq:Vm} è possibile calcolare la frequenza di risonanza attesa del midrange [\ref{sec:resFreq}].
\err{è approssimata}

\begin{equation}
	f_{r} = \frac{1}{2\pi}\sqrt{\frac{1}{L_{m} C_{m}}}  %equazione approssimata
	\label{eq:fr}
\end{equation}

\noindent
Infine, si può ottenere \err{la differenza d fase tra tweeter e woofer alla frequenza di crossover}.

\begin{equation}
	booh non so se farlo
\end{equation}

\section*{Apparato sperimentale e svolgimento}
\prof{ Descrivere l'apparato utilizzato. Allegare una foto o lo schema
	dell'apparato sperimentale (sarà la Fig. 1); la foto deve permettere di
	identificare chiaramente i componenti principali (eventualmente usare delle
	frecce e delle etichette per evidenziarli). Riassumere lo svolgimento delle
	misure e motivare la scelta dei parametri di acquisizione (per esempio:
	l'intervallo di frequenze e la frequenza di acquisizione). }

Per l'analisi in frequenza di ampiezza e fase è stato acquisito ciascun canale
separatamente, con sweep ripetuti. In ogni prova l'analog input \texttt{A1} è
stato collegato al canale di interesse mentre il canale \texttt{A0} è rimasto
collegato al generatore di tensione, utilizzandolo come trigger e riferimento
per la fase. Come range di frequenze è stato scelto: \err{range?}; questo per
evidenziare il punto di crossover ma anche per vedere l'andamento a basse e
alte frequenze. Al fine di massimizzare la frequenza di campionamento, le
tensioni sono state misurate in modalità referenced single ended collegando a
massa il ramo in comune alle resistenze. Ciò ha permesso un campionamento a
$500$ kHz per canale, con $2000$ sample per ciascuna acquisizione, abbastanza
per includere \err{diversi periodi} anche per le frequenze inferiori.

I componenti del circuito sono stati misurati tramilte il multimetro di Elvis
II, i loro valori si trovano in \cref{tab:componenti}.

\noindent
\begin{minipage}[b]{0.6\textwidth}
	\vspace{0pt}
	\centering
	\begin{circuitikz}[scale=1]
		% generatore
		\draw (8,0) --
		(0,0) --
		(0,1.5) to[sinusoidal voltage source]
		(0,3) to[R, l=$R_g$, color=blue, bipoles/resistor/height=0.15, bipoles/resistor/width=0.5]
		(0,4) --
		(0,5.5) --
		(8,5.5);

		\draw(-0,4.5) to[short, -o]
		(-0.6,4.5) node[left]{$v_g$};
		\draw (-0,1) -- (-0.6,1) node[ground]{};

		% woofer
		\draw (3,5.5) --
		(3,4.5) to[L=$L_w$]
		(3,3.5) to[R, l=$R_{Lw}$, color=blue, bipoles/resistor/height=0.15, bipoles/resistor/width=0.5]
		(3,2.5) --
		(3,2) to[R=$R_w$] (3,0);
		\draw (3,2) to [short, -o] (3.5, 2) node[right]{$v_w$};

		% mid
		\draw (5.5,5.5) to [C=$C_m$]
		(5.5,4.25) to [L=$L_m$]
		(5.5,3.25) to [R, l=$R_{Lm}$, color=blue, bipoles/resistor/height=0.15, bipoles/resistor/width=0.5]
		(5.5,2.25) --
		(5.5,2) to[R=$R_m$] (5.5,0);

		\draw (5.5,2) to [short, -o] (6, 2) node[right]{$v_m$};

		%tweeter
		\draw (8,5.5) to[C=$C_t$]
		(8,2) to[R=$R_t$] (8,0);

		\draw (8,2) to
		[short, -o] (8.5, 2) node[right]{$v_t$};
	\end{circuitikz}
	\label{fig:schema_elettrico}
	\captionof{figure}{Schema elettrico del circuito, in blu i componenti non ideali. I punti $v_g, v_w, v_m, v_t$ sono stati collegati agli analog input di Elvis II per le relative misure di tensione.}
\end{minipage}
\begin{minipage}[b]{0.4\textwidth}
	\vspace{0pt}
	\centering
	\begin{alignat*}{2}
		 & R_w    &  & =(3293 \pm 2) \text{ \Omega}      \\
		 & L_w    &  & =(47.2 \pm 0.5) \text{ mH}        \\
		 & R_{Lw} &  & =(120.00 \pm 0.15) \text{ \Omega} \\[1em]
		 & R_m    &  & =(3287 \pm 2) \text{ \Omega}      \\
		 & L_m    &  & =(46.8 \pm 0.5) \text{ mH}        \\
		 & R_{Lm} &  & =(120.70 \pm 0.15) \text{ \Omega} \\
		 & C_m    &  & =(4.76 \pm 0.05) \text{ nF}       \\[1em]
		 & R_t    &  & =(3295 \pm 2) \text{ \Omega}      \\
		 & C_t    &  & =(4.76 \pm 0.05) \text{ nF}       \\[1em]
		 & R_g    &  & =(\err{50}) \text{ \Omega}
	\end{alignat*}
	\captionof{table}{Valori componenti.}
	\label{tab:componenti}
\end{minipage}

\section*{Risultati e discussione}
\prof{ Riportare i risultati più rappresentativi in forma grafica oppure come
	foto delle osservazioni sull'oscilloscopio analogico (se utilizzato). Non è
	necessario riportare tutti i dati. Commentare qualitativamente gli andamenti
	delle grandezze fisiche riportati in forma grafica e/o le forme di riga
	osservate. In tutti i grafici gli assi devono essere chiaramente etichettati e
	le unità di misura devono essere incluse. Quando le incertezze sono note e
	sono visibili sulla scala utilizzata rappresentarle sul grafico come barre di
	errore. Porre particolare attenzione alla leggibilità, utilizzando caratteri
	sufficientemente grandi. Descrivere come sono stati elaborati i dati e
	riportare i risultati numerici (miglior stima ed incertezza), in forma
	tabellare se necessario. Commentare i valori ottenuti. Non è necessario
	riportare il calcolo esplicito delle incertezze (eventualmente usare una
	appendice), ma è importante segnalare se si tratta di risoluzione strumentale,
	di errore casuale oppure di errore sistematico. }

\section*{Conclusioni}
\prof{Conclusioni finali, particolarmente importanti nel caso di risultati
	anomali.}

\appendix
\section{Appendici}
\subsection{Tensioni ai capi delle resistenze}
\label{sec:tensioni}

Ricaviamo per esempio \eqref{eq:Vw} : dalla legge di Kirchhoff simbolica per le
tensioni e correnti abbiamo che

\begin{equation}
	\begin{cases}
		\mathbf{V_{w}} = \mathbf{I_{w}} \mathbb{Z}_{w}                                                                \\
		\mathbf{I_{tot}} = \mathbf{I_{w}} + \mathbf{I_{m}} + \mathbf{I_{t}} = \frac{V_{0}}{R_{g} + \mathbb{Z}_{load}} \\
		\mathbf{I_{w}} \mathbb{Z}_{w} = \mathbf{I_{m}} \mathbb{Z}_{m} = \mathbf{I_{t}} \mathbb{Z}_{t}                 \\
	\end{cases}
\end{equation}

\noindent
da cui si ricava che \err{ha senso ricavarle o possiamo evitare questa appendice?}

\begin{equation}
	\mathbf{V_{w}} = \left( \frac{R_{w}}{\mathbb{Z}_{w}} \right) 
\end{equation}

\begin{equation*}
	\mathbb{Z}_{load}(\omega) = \left(\frac{1}{\mathbb{Z}_{w}(\omega)} + \frac{1}{\mathbb{Z}_{m}(\omega)} + \frac{1}{\mathbb{Z}_{t}(\omega)}\right)^{-1}
\end{equation*}

\begin{equation}
	\left| \mathbf{V_{g}}(\omega) \right| = \left| \frac{\mathbb{Z}_{load}(\omega)}
	{R_{g}+\mathbb{Z}_{load}(\omega)}\right| \mathbf{V}
\end{equation}

\begin{equation}
	\left| \mathbf{V_{w}}(\omega) \right| = \left| \frac{R_{w}}
	{\mathbb{Z}_{w}(\omega)}\right|\left| \mathbf{V_{g}}(\omega) \right| = \left| \frac{R_{w}}
	{R_{w}+R_{Lw} + j \omega L_w}\right|\left| \mathbf{V_{g}}(\omega) \right|
\end{equation}

\begin{equation}
	\left| \mathbf{V_{m}}(\omega) \right| = \left| \frac{R_{m}}
	{\mathbb{Z}_{m}(\omega)}\right|\left| \mathbf{V_{g}}(\omega) \right| = \left| \frac{R_{m}}{R_{m}+R_{Lm} + j (\omega L_m - \frac{1}{\omega C_{m}})} \right| \left| \mathbf{V_{g}}(\omega) \right|
\end{equation}

\begin{equation}
	\left| \mathbf{V_{t}}(\omega) \right| = \left| \frac{R_{t}}
	{\mathbb{Z}_{t}(\omega)}\right|\left| \mathbf{V_{g}}(\omega) \right| = \left| \frac{R_{t}}{R_{t} + j (-\frac{1}{\omega C_{t}})}\right|\left| \mathbf{V_{g}}(\omega) \right|
\end{equation}

\subsection{Fasi delle tensioni}
\label{sec:fasi}

\begin{equation}
	\phi_{w}(\omega) = - \arctan\left(\frac{\omega L_{w}}{R_{w}+R_{Lws}}\right)
\end{equation}

\begin{equation}
	\phi_{m}(\omega) = - \arctan\left(\frac{\omega L_{m} - \frac{1}{\omega C_{m}}}{R_{m}+R_{Lm}}\right)
\end{equation}

\begin{equation}
	\phi_{t}(\omega) = \arctan\left(\frac{1}{\omega R_{t} C_{t}}\right)
\end{equation}

\subsection{Frequenza di crossover}
\label{sec:crossFreq}

\subsection{Frequenza di risonanza}
\label{sec:resFreq}

\subsection{Incertezze}

\end{document}