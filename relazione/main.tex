\documentclass[12pt,italian]{article}
\usepackage[italian]{babel}
\usepackage[a4paper, margin=2cm]{geometry}
\usepackage{amsmath}
\usepackage{circuitikz}
\usepackage{caption}
\usepackage{cleveref}

\newcommand{\prof}[1]{\textcolor{blue}{#1}}
\newcommand{\err}[1]{\textcolor{red}{#1}}

%\usepackage{fontspec} \setmainfont{calibri}
%\usepackage{titlesec}
%\titleformat{\section}
%{\normalfont\normalsize\itshape}{\thesection.}{1em}{}
%\titlespacing{\section}{0pt}{1em}{0pt}


\title{TITOLO}
\author{Enrico Barbuio \\ 0001117553 \and Giacomo Cicala} 
\date{data}

\begin{document}
\maketitle
\section*{Abstract}
\prof{ Lunghezza massima 10 righe, riassumere lo scopo dell'esperimento e i
  risultati principali (osservazioni qualitative e risultati numerici).
  L'abstract non deve contenere riferimenti a quanto riportato nel testo
  (figure, equazioni o tabelle) e deve essere auto - esplicativo. }

\section*{Introduzione}
\prof{ Descrivere i concetti fisici principali e gli obiettivi dell'esperienza.
  Per l'esperienza sui circuiti descrivere il circuito progettato, le
  espressioni algebriche per le grandezze fisiche misurate, il comportamento
  atteso; analogamente per ottica. L'introduzione non è un riassunto, non
  riportare i risultati. }

\section*{Apparato sperimentale e svolgimento}
\prof{ Descrivere l'apparato utilizzato. Allegare una foto o lo schema
  dell'apparato sperimentale (sarà la Fig. 1); la foto deve permettere di
  identificare chiaramente i componenti principali (eventualmente usare delle
  frecce e delle etichette per evidenziarli). Riassumere lo svolgimento delle
  misure e motivare la scelta dei parametri di acquisizione (per esempio:
  l'intervallo di frequenze e la frequenza di acquisizione). }

Per l'analisi in frequenza di ampiezza e fase è stato acquisito ciascun canale
separatamente, con sweep ripetuti. In ogni prova l'analog input $A1$ è stato
collegato al canale di interesse mentre il canale $A0$ è rimasto collegato al
generatore di tensione, utilizzandolo come trigger e riferimento per la fase.
Come range di frequenze è stato scelto: \err{range?}; questo per evidenziare il
punto di crossover ma anche per vedere l'andamento a basse e alte frequenze. Al
fine di massimizzare la frequenza di campionamento, le tensioni sono state
misurate in modalità referenced single ended collegando a massa il ramo in
comune alle resistenze. Ciò ha permesso un campionamento a $500$ kHz per
canale, con $2000$ sample per ciascuna acquisizione, abbastanza per includere
\err{diversi periodi} anche per le frequenze inferiori.

I componenti del circuito sono stati misurati tramilte il multimetro di Elvis
II, i loro valori si trovano in \cref{tab:componenti}.

\noindent
\begin{minipage}[b]{0.6\textwidth}
  \vspace{0pt}
  \centering
  \begin{circuitikz}[scale=1]
    % generatore
    \draw (8,0) --
    (-0,0) to[sinusoidal voltage source]
    (0,5) --
    (8,5);

    \draw(-0,4) to[short, -o]
    (-0.5,4) node[left]{$v_g$};
    \draw (-0,1) -- (-0.5,1) node[ground]{};

    % woofer
    \draw (3,5) to[L=$L_w$] (3,2) to[R=$R_w$] (3,0);
    \draw (3,2) to [short, -o] (3.5, 2) node[right]{$v_w$};

    % mid
    \draw (5.5,5) to [C=$C_m$] (5.5,3.75) to [L=$L_m$] (5.5,2) to[R=$R_m$] (5.5,0);

    \draw (5.5,2) to [short, -o] (6, 2) node[right]{$v_m$};

    %tweeter
    \draw (8,5) to[C=$C_t$]
    (8,2) to[R=$R_t$] (8,0);

    \draw (8,2) to
    [short, -o] (8.5, 2) node[right]{$v_t$};
  \end{circuitikz}
  \label{fig:schema_elettrico}
  \captionof{figure}{Schema elettrico del circuito ideale. I punti $v_g, v_w, v_m, v_t$ sono stati collegati agli analog input di Elvis II per le relative misure di tensione.}
\end{minipage}
\begin{minipage}[b]{0.4\textwidth}
  \vspace{0pt}
  \centering
  \begin{alignat*}{2}
     & R_w    &  & =(3293 \pm 2) \text{ \Omega}      \\
     & L_w    &  & =(47.2 \pm 0.5) \text{ mH}        \\
     & R_{Lw} &  & =(120.00 \pm 0.15) \text{ \Omega} \\[1em]
     & R_m    &  & =(3287 \pm 2) \text{ \Omega}      \\
     & L_m    &  & =(46.8 \pm 0.5) \text{ mH}        \\
     & R_{Lm} &  & =(120.70 \pm 0.15) \text{ \Omega} \\
     & C_m    &  & =(4.76 \pm 0.05) \text{ nF}       \\[1em]
     & R_t    &  & =(3295 \pm 2) \text{ \Omega}      \\
     & C_t    &  & =(4.76 \pm 0.05) \text{ nF}
  \end{alignat*}
  \captionof{table}{Valori componenti.}
  \label{tab:componenti}
\end{minipage}

\section*{Risultati e discussione}
\prof{ Riportare i risultati più rappresentativi in forma grafica oppure come
  foto delle osservazioni sull'oscilloscopio analogico (se utilizzato). Non è
  necessario riportare tutti i dati. Commentare qualitativamente gli andamenti
  delle grandezze fisiche riportati in forma grafica e/o le forme di riga
  osservate. In tutti i grafici gli assi devono essere chiaramente etichettati e
  le unità di misura devono essere incluse. Quando le incertezze sono note e
  sono visibili sulla scala utilizzata rappresentarle sul grafico come barre di
  errore. Porre particolare attenzione alla leggibilità, utilizzando caratteri
  sufficientemente grandi. Descrivere come sono stati elaborati i dati e
  riportare i risultati numerici (miglior stima ed incertezza), in forma
  tabellare se necessario. Commentare i valori ottenuti. Non è necessario
  riportare il calcolo esplicito delle incertezze (eventualmente usare una
  appendice), ma è importante segnalare se si tratta di risoluzione strumentale,
  di errore casuale oppure di errore sistematico. }

\section*{Conclusioni}
\prof{Conclusioni finali, particolarmente importanti nel caso di risultati
  anomali.}
\end{document}